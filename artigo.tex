\documentclass[12pt]{article}
\usepackage{sbc-template}
\usepackage{graphicx,url,wrapfig}
\usepackage{listings}
% indentfirst, wrapfig
\usepackage[brazil]{babel}
%\usepackage[latin1]{inputenc}
\usepackage[utf8]{inputenc}
% UTF-8 encoding is recommended by ShareLaTex
\sloppy
\title{LCD TFT 2.4'}
\author{Jonathan de Oliveira Cardoso\inst{1} Marlon H. Schweigert\inst{1} Mateus S. Tavares\inst{1}}
\address{Departamento de Ciência da Computação\\Universidade do Estado de Santa Catarina (UDESC)\\
  Joinville -- SC -- Brazil
  \email{jonnyydeoliveira@gmail.com, fleyhe0@gmail.com, maseta.mateus@gmail.com}}

\begin{document}
\maketitle
\section{Introdução} \label{sec:relatorio}

\begin{center}
	\includegraphics[width=10cm]{Display_TFT_Texto_Graficos.png}
\end{center}

Bibliotecas utilizadas foram Adafruit\_GFX, Adafruit\_TFTLCD e um wrapper delas(CUFRIEND\_kbv).
Essa biblioteca implementa métodos que escrevem os pixels na tela TFT pelos 6 pinos LCD Data.

\begin{lstlisting}[language=c]
  tft.drawRoundRect(5, 70, 312, 50, 5, WHITE);
  tft.drawRoundRect(255, 70, 62, 50, 5, WHITE);
  tft.setTextColor(GREEN);
  tft.setTextSize(3);
  tft.setCursor(15, 85);
  tft.println("Led Verde");
\end{lstlisting}

Além de escrever pixels diretamente, ele permite desenhar padrões diretamente na tela.
Dessa forma podemos desenhar padrões como retângulos, circunferências, retas, pontos
e até mesmo caracteres.

\section{Pinagem} \label{sec:pin}

\begin{center}
	\includegraphics[width=7cm]{Tabela_Pinos_TFT.png}
\end{center}

Os 3 pinos iniciais (descritos em verde) para alimentação do display. Esse display é alimentado com 3.5 volts. Outra entrada é dedicada a alimentação do leitor de Cartão SD.

O pino A0 referece a sincronização de dados com o display.
Os pinos A1 e D6 são referentes ao posicionamento do do toque na tela.
Os pinos A2 e D7 são referentes ao movimento exercido no dedo.

Os pinos D2, D3, D4, D5, D8 e D9 são referentes as informações Posição(x,y), utilizando o sistema carteziano e Cor(r,g,b), utilizando o padrão vermelho, verde e azul.

\section{Componentes} \label{sec:comp}

O Shield contém componentes para viabilizar a utilização direta do display TFT 2.4' diretamente com o arduino. Isso inclui dois circuítos integrados para transformar 5v para 3.5v, e visse versa.

Além disso, ele já traz uma placa com plugins diretos ao padrão do projeto arduino, transformando o display em um periférico \textit{plug and play}
\begin{center}
	\includegraphics[width=10cm]{mcu_shield.jpg}
\end{center}

\bibliographystyle{sbc}
\bibliography{sbc-template}

%/remover pra avisar referencias nao utilizadas
\nocite{*}

\end{document}
